% !TEX program = xelatex
\documentclass{ctexart}

\usepackage{amsmath}
\usepackage{physics}
\usepackage{tikz}
\usepackage{graphics}
\usepackage{array}
\usepackage{xcolor}
\usepackage{amssymb}
\usepackage{listing}
\usepackage{amsfonts}
\usepackage{mathrsfs}

\newtheorem{definition}{definition}
\newtheorem{theorem}{theorem}
\newtheorem{exercise}{exercise}

\title{每日一题\\{\small 线性代数-1}}

\date{2023.1.3}

\begin{document}
    \maketitle {}

    我们来学习基础的线性代数与矩阵的内容。

    \begin{definition}
        [矩阵] 矩阵\(A\)是\(m\)行\(n\)列的数, 第i行j列的数可以表示为\(A_{ij}\)。
    \end{definition}

    矩阵不是凭空诞生的, 其背后有数学意义, 最简单的, 是矩阵可以表示线性变换 (线性代表\(y=ax\)) 。

    \begin{definition}
        [线性变换-朴素] 线性变换就是长得像下面一样的变换
        \begin{equation}
            \left \{
            \begin{aligned}
                y_1 &= a_{11}x_1 + a_{12}x_2 + \cdots + a_{1n}x_n \\
                y_2 &= a_{21}x_1 + a_{22}x_2 + \cdots + a_{2n}x_n \\
                \vdots \\
                y_m &= a_{m1}x_1 + a_{m2}x_2 + \cdots + a_{mn}x_n
            \end{aligned}
            \right .
        \end{equation}
        可以发现, 每个y对于每个x都是线性的, 所以称其为线性变换。
        
        将上面的式子记成矩阵的形式, 就是
        \begin{equation}
            \label {eq:线性变换的矩阵表示}
            \begin{bmatrix}
                y_1 \\
                y_2 \\
                \vdots \\
                y_m
            \end{bmatrix} = \begin{bmatrix}
                a_{11} & a_{12} & \cdots & a_{1n} \\
                a_{21} & a_{22} & \cdots & a_{2n} \\
                \vdots & \vdots & \ddots & \vdots \\
                a_{m1} & a_{m2} & \cdots & a_{mn}
            \end{bmatrix} \begin{bmatrix}
                x_1 \\
                x_2 \\
                \vdots \\
                x_n
            \end{bmatrix}
        \end{equation}
        其中, 被方括号框起来的是矩阵, 其分量就是对应的数, 只有一行的矩阵称作行向量, 
        只有一列的矩阵称作列向量。
    \end{definition}

    我们来考虑两个线性变换叠加的例子。

    \begin{equation}
        \label {eq:线性变换叠加1}
        \left \{
        \begin{aligned}
            y_1 &= a_{11}x_1 + a_{12}x_2 + \cdots + a_{1n}x_n \\
            y_2 &= a_{21}x_1 + a_{22}x_2 + \cdots + a_{2n}x_n \\
            \vdots \\
            y_m &= a_{m1}x_1 + a_{m2}x_2 + \cdots + a_{mn}x_n
        \end{aligned}
        \right .
    \end{equation}

    \begin{equation}
        \label {eq:线性变换叠加2}
        \left \{
        \begin{aligned}
            z_1 &= b_{11}y_1 + b_{12}y_2 + \cdots + b_{1m}y_m \\
            z_2 &= b_{21}y_1 + b_{22}y_2 + \cdots + b_{2m}y_m \\
            \vdots \\
            z_k &= b_{n1}y_1 + b_{n2}y_2 + \cdots + b_{nm}y_m
        \end{aligned}
        \right .
    \end{equation}

    把 (\ref{eq:线性变换叠加1}) 式带入 (\ref{eq:线性变换叠加2}) 式中, 就得到了

    \begin{equation}
        \label {eq:线性变换叠加3}
        \left \{
        \begin{aligned}
            z_1 &= (\sum_{i=1}^{m} b_{1i}a_{i1})x_1 + (\sum_{i=1}^{m} b_{1i}a_{i2})x_2 + \cdots + (\sum_{i=1}^{m} b_{1i}a_{in})x_n \\
            z_2 &= (\sum_{i=1}^{m} b_{2i}a_{i1})x_1 + (\sum_{i=1}^{m} b_{2i}a_{i2})x_2 + \cdots + (\sum_{i=1}^{m} b_{2i}a_{in})x_n \\
            \vdots \\
            z_k &= (\sum_{i=1}^{m} b_{ki}a_{i1})x_1 + (\sum_{i=1}^{m} b_{ki}a_{i2})x_2 + \cdots + (\sum_{i=1}^{m} b_{ki}a_{in})x_n
        \end{aligned}
        \right .
    \end{equation}

    我们认为矩阵乘法等同于线性变换的叠加, 由此得到矩阵乘法的定义。

    \begin{definition}
        [矩阵乘法] 矩阵乘法定义是
        \begin{equation}
            \begin{bmatrix}
                a_{11} & a_{12} & \cdots & a_{1n} \\
                a_{21} & a_{22} & \cdots & a_{2n} \\
                \vdots & \vdots & \ddots & \vdots \\
                a_{m1} & a_{m2} & \cdots & a_{mn}
            \end{bmatrix} \begin{bmatrix}
                b_{11} & b_{12} & \cdots & b_{1k} \\
                b_{21} & b_{22} & \cdots & b_{2k} \\
                \vdots & \vdots & \ddots & \vdots \\
                b_{n1} & b_{n2} & \cdots & b_{nk}
            \end{bmatrix} = \begin{bmatrix}
                \sum_{i=1}^{n} a_{1i}b_{i1} & \sum_{i=1}^{n} a_{1i}b_{i2} & \cdots & \sum_{i=1}^{n} a_{1i}b_{ik} \\
                \sum_{i=1}^{n} a_{2i}b_{i1} & \sum_{i=1}^{n} a_{2i}b_{i2} & \cdots & \sum_{i=1}^{n} a_{2i}b_{ik} \\
                \vdots & \vdots & \ddots & \vdots \\
                \sum_{i=1}^{n} a_{mi}b_{i1} & \sum_{i=1}^{n} a_{mi}b_{i2} & \cdots & \sum_{i=1}^{n} a_{mi}b_{ik}
            \end{bmatrix}
        \end{equation}
        
        采用爱因斯坦求和标记 (出现两次的指标就是求和) 并展开成分量, 我们可以写成
        \begin{equation}
            \label {eq:矩阵乘法}
            {(AB)}_{ij} = A_{ik}B_{kj}
        \end{equation}
        的形式 (如你所见, 写成分量式漂亮很多) 。
    \end{definition}

    可以验证, (\ref{eq:线性变换的矩阵表示}) 式满足矩阵乘法的定义。

    \begin{definition}
        [矩阵加法] 相对的, 描述如下线性变换的变化, 就是矩阵加法。
        \begin{equation}
            \label {eq:矩阵加法}
            \left \{
            \begin{aligned}
                y_1 &= a_{11}x_1 + a_{12}x_2 + \cdots + a_{1n}x_n \\
                y_2 &= a_{21}x_1 + a_{22}x_2 + \cdots + a_{2n}x_n \\
                \vdots \\
                y_m &= a_{m1}x_1 + a_{m2}x_2 + \cdots + a_{mn}x_n
            \end{aligned}
            \right .
        \end{equation}
        \begin{equation}
            \label {eq:矩阵加法2}
            \left \{
            \begin{aligned}
                z_1 &= b_{11}x_1 + b_{12}x_2 + \cdots + b_{1n}x_n \\
                z_2 &= b_{21}x_1 + b_{22}x_2 + \cdots + b_{2n}x_n \\
                \vdots \\
                z_m &= b_{m1}x_1 + b_{m2}x_2 + \cdots + b_{mn}x_n
            \end{aligned}
            \right .
        \end{equation}
        \begin{equation}
            \label {eq:矩阵加法3}
            \left \{
            \begin{aligned}
                y_1 + z_1 &= (a_{11} + b_{11})x_1 + (a_{12} + b_{12})x_2 + \cdots + (a_{1n} + b_{1n})x_n \\
                y_2 + z_2 &= (a_{21} + b_{21})x_1 + (a_{22} + b_{22})x_2 + \cdots + (a_{2n} + b_{2n})x_n \\
                \vdots \\
                y_m + z_m &= (a_{m1} + b_{m1})x_1 + (a_{m2} + b_{m2})x_2 + \cdots + (a_{mn} + b_{mn})x_n
            \end{aligned}
            \right .
        \end{equation}
        由此得到矩阵加法的定义 (此处 ij 出现三次不求和) 。 
        \begin{equation}
            {(A + B)}_{ij} = A_{ij} + B_{ij}
        \end{equation}
    \end{definition}

    \begin{definition}
        [矩阵幂次] 定义是显然的。
        \begin{equation}
            A^n = A \cdot A \cdot \cdots \cdot A \text {(n个A)}\\
        \end{equation}
    \end{definition}

    \begin{exercise}
        \begin{itemize}
            \item 证明矩阵加法与乘法的结合律
                    \begin{equation}
                        A + (B + C) = (A + B) + C
                    \end{equation}
                    \begin{equation}
                        A(BC) = (AB)C
                    \end{equation}
            \item 证明矩阵加法与乘法的分配律
                    \begin{equation}
                        A(B + C) = AB + AC
                    \end{equation}
                    \begin{equation}
                        (A + B)C = AC + BC
                    \end{equation}
            \item 使用矩阵幂次的形式表达斐波那契数列的通项。
                    \begin{equation}
                        \begin{aligned}
                            a_n = a_{n-1} + a_{n-2} \\
                            a_{n-1} = a_{n-1}
                        \end{aligned}
                    \end{equation}
            \item 考察数列, 用矩阵幂次表示其通项。
                    \begin{equation}
                        a_{n+1} = \frac {A a_n + B}{C a_n + D}
                    \end{equation}
                    提示: 带入\(a_{n-1}\)并观察。
        \end{itemize}
    \end{exercise}
\end{document}

