% !TEX program = xelatex
\documentclass {ctexart}

\usepackage {amsmath}
\usepackage {physics}
\usepackage {tikz}
\usepackage {graphics}
\usepackage {array}
\usepackage {xcolor}
\usepackage {amssymb}
\usepackage {listing}

\title {每日一题\\{\small 傅立叶变换-1}}

\date{2022.12.30}

\begin{document}
    \maketitle {}

    接下来几天, 我们将学习傅立叶变换 (只需要高数知识) 。

    考虑一定范围\([a,b]\)内的连续函数, 定义运算符\(\cdot\), 使得\(f(x) \cdot g(x) = \int_a^b f(x)g(x) \dd x\)。

    接下来我们考虑两组函数\(f_n(x),f^m(x)\), 满足\(f_n(x) \cdot f^m(x) = \delta_n^m\)。其中\(\delta_n^m\)在\(m=n\)时为1, 在\(m \neq n\)时为0。

    我们希望把某个函数\(f(x)\)分解成这组函数的线性组合, 即\(f(x) = \sum_{n=1}^\infty c_n f_n(x)\), 其中\(c_n\)是常数, 只需将\(f(x)\)分别点乘上\(f^n(x)\)就能得到\(c_n\)。
    
    我们取

    \[
        f_n(x) = e^{\frac{2n\pi}{b-a} i x}
    \]

    则

    \[
        f^m(x) = \frac{1}{b-a} e^{-\frac{2m}{b-a} i x}
    \]

    此时, 有\(f_n(x)f^m(x) = \delta_n^m\), 从而得到傅立叶变换:

    \begin{equation}
        f(x) = \frac{1}{b-a} \sum_{n=-\infty}^{\infty} e^{\frac{2n\pi}{b-a} i x} \int_a^b f(X)e^{-\frac{2n\pi}{b-a} i X} \dd X
    \end{equation}
        

    \begin{itemize}
        \item 对函数\(f(x) = x, x \in [0,2\pi] \)进行傅立叶变换。
        \item 对函数\(f(x) = x^2, x \in [0,2\pi] \)进行傅立叶变换, 并把右式求导检查你的结果。
        \item 反过来将右边积分, 计算\(\sum_{n=1}^{\infty} \frac{1}{n^2}\), \(\sum_{n=1}^{\infty} \frac{1}{n^4}\), \(\sum_{n=1}^{\infty} \frac{1}{n^6}\)
        \item (补充) 事实上上面过程隐含了任何一个连续函数都能傅立叶展开, 证明傅立叶展开的求和式满足
        \[
            \lim_{k \rightarrow \infty} \int_a^b (f(x) - \frac{1}{b-a}\sum_{n=-k}^{k} e^{\frac{2n\pi}{b-a} i x} \int_a^b f(X)e^{-\frac{2n\pi}{b-a} i X} \dd X )^2 \dd x
        \]
    \end{itemize}

\end{document}

