% !TEX program = xelatex
\documentclass{ctexart}

\usepackage{amsmath}
\usepackage{physics}
\usepackage{tikz}
\usepackage{graphics}
\usepackage{array}
\usepackage{xcolor}
\usepackage{amssymb}
\usepackage{listing}
\usepackage{amsfonts}
\usepackage{mathrsfs}

\newtheorem{definition}{definition}
\newtheorem{theorem}{theorem}
\newtheorem{exercise}{exercise}
\newtheorem{example}{example}

\title{每日一题\\{\small 线性代数-2}}

\date{2023.1.4}

\begin{document}
    \maketitle {}

    \begin{definition}
        [矩阵转置]
        \begin{equation}
            A^T_{ij} = A_{ji}
        \end{equation}
    \end{definition}
    如你所见, 矩阵转置就是把矩阵沿着\(45^\circ\)对角线翻转.
    定义了矩阵转置以后, 如下性质读者自证不难.
    \begin{exercise}
        证明:
        \begin{equation}
            \begin{aligned}
                A = {(A^T)}^T \\
                {(A+B)}^T = A^T + B^T \\
                {(AB)}^T = B^{T}A^{T}
            \end{aligned}
        \end{equation}
    \end{exercise}
    我们把昨天的线性变换全体转置得到.
    \begin{equation}
        \begin{bmatrix}
            y_1 & y_2 & \cdots & y_n
        \end{bmatrix} = \begin{bmatrix}
            x_1 & x_2 & \cdots & x_m
        \end{bmatrix} \begin{bmatrix}
            a_{11} & a_{21} & \cdots & a_{n1} \\
            a_{12} & a_{22} & \cdots & a_{n2} \\
            \vdots & \vdots & \ddots & \vdots \\
            a_{1m} & a_{2m} & \cdots & a_{nm}
        \end{bmatrix}
    \end{equation}

    矩阵转置暂时用不到, 但我们先保留它的定义.

    现在我们已经把线性变换用矩阵表示了, 得到 y 相对于 x 的关系, 但是我们还不知道 x 相对于 y 的关系.

    \begin{definition}
        [逆矩阵] 假定 \(A\) 是一个方阵(行数等于列数), 那么如果存在一个矩阵记作 \(A^{-1}\), 使得
        \begin{equation}
            A^{-1}A = I
        \end{equation}
        I 是单位矩阵, 也就是对角线上全是 1, 其他地方全是 0 的方阵, 有时也用 E 表示.
        逆矩阵的唯一性和存在性将在学习行列式后讨论.
    \end{definition}

    带入之前的线性变换, 我们得到 (自行带入验证不难) 

    \begin{equation}
        \begin{bmatrix}
            x_1 \\
            x_2 \\
            \vdots \\
            x_n
        \end{bmatrix} = A^{-1} \begin {bmatrix}
            y_1 \\
            y_2 \\
            \vdots \\
            y_n
        \end{bmatrix}
    \end{equation}

    求逆矩阵最常见的方法就是高斯消元法, 俗称加加减减出奇迹.

    \begin{example}
        \begin{equation}
            \begin{bmatrix}
                1 & 2 \\
                3 & 4
            \end{bmatrix}
        \end{equation}

        也就是解方程组

        \begin{equation}
            \begin{aligned}
                y_1 &= x_1 + 2x_2  \\
                y_2 &= 3x_1 + 4x_2
            \end{aligned}
        \end{equation}

        上面乘以三倍减去下面, 得到

        \begin{equation}
            \begin{aligned}
                3y_1 - y_2 = 2x_2 \\
                y_2 = 3x_1 + 4x_2
            \end{aligned}
        \end{equation}

        下面乘以二分之一减去上面, 得到

        \begin{equation}
            \begin{aligned}
                x_1 &= y_2 - 2y_1 \\
                x_2 &= \frac{3y_1 - y_2}{2}
            \end{aligned}
        \end{equation}

        从而, 我们有

        \begin{equation}
            \begin{bmatrix}
                -2 & 1 \\
                \frac{3}{2} & -\frac{1}{2}
            \end{bmatrix} \begin{bmatrix}
                1 & 2 \\
                3 & 4
            \end{bmatrix} = \begin{bmatrix}
                1 & 0 \\
                0 & 1
            \end{bmatrix}
        \end{equation}

        我们来仔细观察上述结果, 写成矩阵的形式.
        
        \begin{equation}
            \begin{bmatrix}
                1 & 0 & 1 & 2 \\
                0 & 1 & 3 & 4
            \end{bmatrix}
        \end{equation}

        \begin{equation}
            \begin{bmatrix}
                3 & -1 & 0 & 2 \\
                0 & 1 & 3 & 4
            \end{bmatrix}
        \end{equation}


        \begin{equation}
            \begin{bmatrix}
                3 & -1 & 0 & 2 \\
                -6 & 3 & 3 & 0
            \end{bmatrix}
        \end{equation}


        \begin{equation}
            \begin{bmatrix}
                -2 & 1 & 1 & 0 \\
                \frac{2}{3} &-\frac{1}{2} & 0 & 1
            \end{bmatrix}
        \end{equation}

        就是通过一些行的加加减减得到逆矩阵.
    \end{example}
    
    \begin{exercise}
        计算任意二 (三) 阶方阵的逆矩阵, 并指出它们可逆的条件 (不等式左边就是行列式).
    \end{exercise}

    \begin{exercise}
        证明如下性质:
        \begin{equation}
            {(AB)}^{-1} = B^{-1}A^{-1}
        \end{equation}
        \begin{equation}
            {(A^T)}^{-1} = {(A^{-1})}^T
        \end{equation}
    \end{exercise}
\end{document}

