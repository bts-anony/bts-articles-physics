\documentclass{ctexart}

\usepackage {amsmath}
\usepackage {physics}
\usepackage {tikz}
\usepackage {graphics}
\usepackage {array}
\usepackage {xcolor}
\usepackage {amssymb}
\usepackage {listing}

\title {每日一题 \\ {\small 三次反比力}}
\date{2022.12.28}

\begin{document}
    \maketitle{}



    考虑引力场\(U = \frac{k}{r^2}\)的情况, 显然有角动量\(\mu = \vec{r} \times \vec{p} \)和能量\(E = \frac{k}{r^2} + \frac{1}{2}m v^2 \)为守恒量。
    \begin{itemize}
        \item 求证I是守恒量。
            \begin{equation}
                I = 2Et - \vec{r} \cdot \vec{p}
            \end{equation}
        \item 利用上述守恒量说明轨道为圆, 发散, 收敛的时候, 能量的正负号。
        \item 导出r与t的关系, 取\(I=0\)时\(t=0\)。
        \item 导出\(\theta\)与t的关系, 取\(I=0\)时\(t=0\)。
        \item 导出轨道方程。
    \end{itemize}
    
\end{document}


