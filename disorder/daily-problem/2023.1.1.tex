% !TEX program = xelatex
\documentclass {ctexart}

\usepackage {amsmath}
\usepackage {physics}
\usepackage {tikz}
\usepackage {graphics}
\usepackage {array}
\usepackage {xcolor}
\usepackage {amssymb}
\usepackage {listing}
\usepackage{amsfonts}
\usepackage{mathrsfs}

\title {每日一题\\{\small 傅立叶变换-3}}

\date{2023.1.1}

\begin{document}
    \maketitle {}

    考虑无穷大正方形电阻网络\((0,0)\)与\((p,q)\)两点的等效电阻。
    \begin{itemize}
        \item 设出每个点的电压\(\phi(x,y)\), 并列出电压满足的方程。
        \item 构造函数, 并利用上一问方程计算出该函数所满足的方程。
                \begin{equation}
                    P(x,y) = \sum_{m=-\infty}^{\infty}\sum_{n=-\infty}^{\infty} \phi(m,n) e^{imx+iny}
                \end{equation}
        \item 解出函数, 并利用傅立叶变换计算出电阻。
        \item 类似的, 将上述公式推导到三维情况。
    \end{itemize}
\end{document}

