\documentclass{ctexart}

\usepackage{amsmath}
\numberwithin{equation}{subsection}
\newtheorem{theorem}    {定理}
\numberwithin{theorem}{subsection}
\newtheorem{definition} {定义}
\numberwithin{definition}{subsection}
\newtheorem{proof}      {证明}
\numberwithin{proof}{subsection}
\newtheorem{lemma}      {引理}
\numberwithin{lemma}{subsection}
\newtheorem{example}    {例子}
\numberwithin{example}{subsection}
\newtheorem{remark}     {备注}
\numberwithin{remark}{subsection}
\newtheorem{corollary}  {推论}
\numberwithin{corollary}{subsection}
\newtheorem{exercise}   {习题}
\numberwithin{exercise}{subsection}
\newtheorem{problem}    {问题}
\numberwithin{problem}{subsection}
\newtheorem{question}   {习题}
\numberwithin{question}{section}
\newtheorem{method}     {方法}
\numberwithin{method}{subsection}
\usepackage{amssymb}
\usepackage{physics}


\begin{document}
    \section{运动学}
    \subsection{我们所考虑的对象}
    运动学顾名思义, 是描述物体运动的学科, 物体的运动, 就是物体在空间中的过程。

    空间是由一系列点构成的, 就如同集合是由一个个元素构成的, 他们都是我们在研究的
    重要的数学对象, 我们把这一系列点记作$\mathbb{M}$。

    一个运动过程, 就是点在空间中的运动, 记作一个函数$S(t)$。
    
    \begin{equation}
        \label{eq:1.1 S(t)}
        S(t):\mathbb{R} \rightarrow \mathbb{M}
    \end{equation}

    对于一个常见的空间, 很容易发现它周围也有一些点, 基于这些点, 我们可以给出在
    空间$\mathbb{M}$上的度量, 拓扑等结构, 这里不赘述, 但是我们先假定这个空间在一定程
    度上是光滑的。

    于是, 在一块较小的空间内, 可以在空间$\mathbb{M}$上赋予一个坐标系(双射)。
    \begin{equation}
        \Omega(q_i):\mathbb{R}^n \rightarrow \mathbb{M}
    \end{equation}
    \begin{equation}
        \Omega^{-1}(P):\mathbb{M} \rightarrow \mathbb{R}^n
    \end{equation}

    则$\mathbb{M}$被称为n维流形。

    一段很长的过程可以认为是一群很短的过程的组合, 我们只考虑全运动过程
    发生在较小一块空间内的情况, 于是\ref{eq:1.1 S(t)}可以写成
    \begin{equation}
        S(t):\mathbb{R} \rightarrow \mathbb{R}^n
    \end{equation}

    既然被称为物体的运动, 物体随时间的变化, 也就是我们要研究的重要问题之一。
    物体在各处都可能往临近的点运动, 描述这个运动, 我们自然而然引入了切从的概念。

    切从$T\mathbb{M}$为一个描述$\mathbb{M}$上每个点的切向量 (微分性质) 的流形。

    假设$\mathbb{M}$是一个n维流形, 则$T\mathbb{M}$在$\mathbb{M}$上每个
    点都形成了一个线性空间, 总体上构成一个2n维流形, 称为切从。

    我们将上面的$\Omega$也拓展到切从之上。

    \begin{equation}
        \Omega(q_i,\dot{q_i}):\mathbb{R}^n\times\mathbb{R}^n \rightarrow T\mathbb{M}
    \end{equation}
    \begin{equation}
        \Omega^{-1}(P):T\mathbb{M} \rightarrow \mathbb{R}^n\times\mathbb{R}^n
    \end{equation}

    我们来考虑坐标系转换的相关事宜, 假定在这块区域内, 有q,Q两个坐标系, 他们的关系
    由下述式子给定。

    \begin{equation}
        q_i = q_i(Q_i)
    \end{equation}
    \begin{equation}
        Q_i = Q_i(q_i)
    \end{equation}

    $\mathbb{M}$上的坐标转换是显然的, 是由上式给出的, 但是给出上式以后, 
    $T\mathbb{M}$上的坐标转换就不那么自由了, 我们考虑一个函数$S(t)$。

    \begin{equation}
        S(t) = (q_i(t)) = (q_i(Q_j(t)))
        \footnotemark
    \end{equation}
    两边求导, 下式是显然的: 

    \begin{equation}
        \dot{S}(t) = (\dot{q_i}(t)) = (\frac{\partial q_i}{\partial Q_j}\dot{Q_j}(t))
    \end{equation}

    上式中使用了爱因斯坦求和标记, 重复指标j被求和掉了, 补充回去会得到:

    \begin{equation}
        \dot{S}(t) = (\dot{q_i}(t)) = (\sum_{j=1}^n\frac{\partial q_i}{\partial Q_j}\dot{Q_j}(t))
    \end{equation}
    以后爱因斯坦求和标记均省略不写。

    由于$S(t)$的任意性, 必须有$T\mathbb{M}$上一个矢量$(q_i,\dot{q_i})$在另一组坐标下是

    \begin{equation}
        (Q_i,\dot{Q_i}) = (Q_i,\frac{\partial Q_i}{\partial q_j}\dot{q_j})
    \end{equation}

    相当于线性空间换了一组基矢量 (上文所说线性空间便是由此而来) 。

    \footnotetext{这里用()表示向量。}

    \subsection{我们所处的三维空间}

    现在我们来考虑一些更具体的内容, 那就是我们所处在的这个三维空间。早在笛卡尔时代
    就提出过用三个实数来表示空间中的一个点的做法, 同样的, 我们把三维空间(记作
    $A_3$), 选择一个给定原点和三个个给定方向, 可以用三个实数描述坐标。

    \begin{equation}
        \Omega : (x,y,z) \mapsto P
    \end{equation}

    由于三维空间是平直的, 所以每个点的切向量都存在简单的对应关系, 
    又按$\vec{a},\vec{b}$的顺序与按照$\vec{b},\vec{a}$的顺序平移(交换性)
    结果是相同的。所以对于$A_3$上任意两个点, 可以把两个点的相对位置
    (从一个点开始, 以一个切向量运行为1得到另一个点)转换成线性空间
    $\mathbb{R}^3$上的对象。

    在这个$A_3$中, 两个点的相对位置加减是有意义的, 但是两个点坐标直接加减是
    无意义的, 相对位置与点相加也是有意义的。

    三维直角坐标系被称作笛卡尔坐标系$(x,y,z)$。

    当然, 我们也可以在三维空间中定义别的坐标系比如球坐标系$(R,\phi,\theta)$
    \begin{equation}
        \begin{cases}
            R = \sqrt{x^2 + y^2 +z^2} \\
            \phi = \arctan{\frac{y}{x}} \\
            \theta = \arccos{\frac{z}{\sqrt{x^2 + y^2 +z^2}}}
        \end{cases}
    \end{equation}

    \begin{equation}
        \begin{cases}
            x = R\sin{\theta}\cos{\phi} \\
            y = R\sin{\theta}\sin{\phi} \\
            z = R\cos{\theta}
        \end{cases}
    \end{equation}

    柱坐标系$(R,\phi,z)$

    \begin{equation}
        \begin{cases}
            R = \sqrt{x^2 + y^2} \\
            \phi = \arctan{\frac{y}{x}} \\
            z = z
        \end{cases}
    \end{equation}

    \begin{equation}
        \begin{cases}
            x = R\cos{\phi} \\
            y = R\sin{\phi} \\
            z = z
        \end{cases}
    \end{equation}
    

    上述关系给定了坐标的变换, 由此, 切从上坐标变换也就可以导得了。

    \begin{equation}
        \begin{cases}
            \dot{R} = \frac{x}{\sqrt{x^2 + y^2 + z^2}} \dot{x} + \frac{y}{\sqrt{x^2 + y^2 + z^2}}\dot{y} + \frac{z}{\sqrt{x^2 + y^2 + z^2}}\dot{z} \\
            \dot{\phi} = \frac{y}{x^2 + y^2}\dot{x} - \frac{x}{x^2 + y^2}\dot{y} \\
            \dot{\theta} = \frac{xz}{\sqrt{x^2 + y^2}(x^2+y^2+z^2)}\dot{x} + \frac{yz}{\sqrt{x^2 + y^2}(x^2+y^2+z^2)}\dot{y} - \frac{\sqrt{x^2 + y^2}}{x^2+y^2+z^2}\dot{z}
        \end{cases}
    \end{equation}
        
    \begin{equation}
        \begin{cases}
            \dot{x} = R\sin{\theta}\cos{\phi}\dot{R} + R\cos{\theta}\cos{\phi}\dot{\theta} - R\sin{\theta}\sin{\phi}\dot{\phi} \\
            \dot{y} = R\sin{\theta}\sin{\phi}\dot{R} + R\cos{\theta}\sin{\phi}\dot{\theta} + R\sin{\theta}\cos{\phi}\dot{\phi} \\
            \dot{z} = R\cos{\theta}\dot{R} - R\sin{\theta}\dot{\theta}
        \end{cases}
    \end{equation}

    \begin{equation}
        \begin{cases}
            \dot{R} = \frac{x}{\sqrt{x^2 + y^2}}\dot{x} + \frac{y}{\sqrt{x^2 + y^2}}\dot{y} \\
            \dot{\phi} = \frac{y}{x^2 + y^2}\dot{x} - \frac{x}{x^2 + y^2}\dot{y} \\
            \dot{z} = \dot{z}
        \end{cases}
    \end{equation}

    \begin{equation}
        \begin{cases}
            \dot{x} = R\cos{\phi}\dot{R} - R\sin{\phi}\dot{\phi} \\
            \dot{y} = R\sin{\phi}\dot{R} + R\cos{\phi}\dot{\phi} \\
            \dot{z} = \dot{z}
        \end{cases}
    \end{equation}
    
    \subsection{三维空间的旋转}

    假设三维中有一个物体, 可以绕定轴旋转, 很容易找到一个参数$\theta$代表其旋转角,
    于是就能对定轴旋转有坐标系。

    假设固定的轴是$z$轴, 那么旋转将会带来一个坐标的改变:

    \begin{equation}
        \begin{bmatrix}
            x^\prime \\ y^\prime \\ z^\prime
        \end{bmatrix}
        = \begin{bmatrix}
            \cos\theta & \sin\theta & 0 \\
            -\sin\theta & \cos\theta & 0 \\
            0 & 0 & 1
        \end{bmatrix}
        \begin{bmatrix}
            x \\ y \\ z
        \end{bmatrix}
    \end{equation}

    我们指出, 旋转就是应用于空间上的变换, 它保持原点不变(假定绕原点旋转), 
    保持任意两个点距离不变, 把旋转代表的变换记作$R$。

    可以证明, 该变换是线性的, 于是所有旋转$R$都可以写成一个矩阵的形式, 矩阵
    保距离 (由内积定义) 不变, 就要求把单位向量映射成单位向量, 把正交向量映射
    成正交向量。

    \begin{equation}
        R \in O(3) = \{A \in Matrix(3) | A^T A = I\}
    \end{equation}

    既然旋转是映射, 旋转自然也能复合, 复合的结果就是矩阵的乘法。
    
    一组旋转的连续性可以由$Matrix(3)$作为$\mathbb{R}^9$上的连续来简单定义,
    引入连续性质后, 我们发现$O(3)$中的元素可以分为两类, 一类行列式为1, 一类
    行列式为-1。($det(I) = 1 = det(AA^T) = det(A) ^2$), 同一类任意两个旋转
    可以用一组连续的旋转(旋转空间中的对象)相关联, 称作联通, 不同类的旋转不联通。
    
    但是考察将我们手边的物体进行旋转, 必然是从一种状态连续旋转到另一种状态, 
    这种旋转是行列式为1的旋转, 不难发现, 这样旋转在$O(3)$中是封闭的。

    \begin{equation}
        R \in SO(3) = \{A \in O(3) | det(A) = 1\}
    \end{equation}

    与之相对, 行列式为-1的矩阵代表空间的反演变换, 需要两次才能回归普通的旋转。

    考虑$R$在矢量上的作用, 

    \begin{equation}
        \begin{bmatrix}
            x \\ y \\ z
        \end{bmatrix}^\prime = R \begin{bmatrix}
            x \\ y \\ z
        \end{bmatrix}
    \end{equation}
    \begin{equation}
        \begin{bmatrix}
            x & y & z
        \end{bmatrix}^\prime = \begin{bmatrix}
            x & y & z
        \end{bmatrix} R^T
    \end{equation}

    我们继续来考虑旋转空间上的代数, 先引入一些数学概念。

    (群) 设G是一个集合, 有一个二元运算\(\cdot\)满足:
    \begin{enumerate}
        \item (封闭性) 任意两个元素\(g_1, g_2 \in G\)都有\(g_1 \cdot g_2 \in G\)
        \item (结合律) 任意三个元素\(g_1, g_2, g_3 \in G\)都有
            \(g_1 \cdot (g_2 \cdot g_3) = (g_1 \cdot g_2) \cdot g_3\)
        \item (含幺性) 存在一个元素\(e \in G\)满足\(g \cdot e = e \cdot g = g\)
        \item (含逆性) 对于任意一个元素\(g \in G\), 存在一个元素\(g^{-1} \in G\)
            满足\(g \cdot g^{-1} = g^{-1} \cdot g = e\)
    \end{enumerate}
    我们称G为群, 运算\(\cdot\)为群的运算。

    群的例子有整数以加法作为群的运算, 非负有理数上以乘法作为群的运算, n个元素上
    的置换等。

    如果一个群运算是交换的, 也就是说对于任意两个元素\(g_1, g_2 \in G\),
    \(g_1 \cdot g_2 = g_2 \cdot g_1\), 我们称这个群为交换群(Abel 群)。

    不难发现, 上述所说的旋转群是一个群, 但不是一个Abel群。

    $SO(3)$和一般的群有所特殊, $SO(3)$是定义在一块连续区域上的群, 那么势必
    要考虑无穷小的群元, 给定光滑函数$R(t)$。

    \begin{equation}
        \begin{bmatrix}
            x \\ y \\ z
        \end{bmatrix}^\prime = R(t) \begin{bmatrix}
            x \\ y \\ z
        \end{bmatrix}
    \end{equation}

    \begin{equation}
        \begin{bmatrix}
            x \\ y \\ z
        \end{bmatrix}^\prime = (R(t)R(\tau)^T) R(\tau) \begin{bmatrix}
            x \\ y \\ z
        \end{bmatrix}
    \end{equation}

    当$\tau$无限接近于$t$时, 无穷小的旋转由矩阵$R(t + \dd t)R(t)^T$
    给出。另一方面, 两边转置, 有

    \begin{equation}
        \begin{bmatrix}
            x & y & z
        \end{bmatrix}^\prime = \begin{bmatrix}
            x & y & z
        \end{bmatrix} R(\tau)^T R(\tau) R(t)^T
    \end{equation}

    得到$R(t) R(t + \dd t)^T$, 考虑到旋转无穷小, 该矩阵和$I$仅差一个$\dd t$
    小量。

    \begin{equation}
        R(t + \dd t)R(t)^T = I + \Omega \dd t 
    \end{equation}

    考虑到左边属于$SO(3)$, 有:

    \begin{equation}
        (I + \Omega \dd t)(I + \Omega \dd t)^T = I
    \end{equation}

    \begin{equation}
        \Omega + \Omega^T = 0
    \end{equation}

    即$\Omega$是一个反对称矩阵, 有三个自由度, 可以和矢量对应起来。

    \begin{equation}
        \Omega = \begin{bmatrix}
            0 & -\omega_3 & \omega_2 \\
            \omega_3 & 0 & -\omega_1 \\
            -\omega_2 & \omega_1 & 0
        \end{bmatrix} \leftrightarrow \begin{bmatrix}
            \omega_1 \\ \omega_2 \\ \omega_3
        \end{bmatrix}
    \end{equation}

    接下来, 我们来考虑转动坐标系在转动变换$U$下的变换。

    \begin{equation}
        R^\prime = U R U^T
    \end{equation}

    同样的, 有:

    \begin{equation}
        \Omega^\prime = U \Omega U^T
    \end{equation}

    \subsection{张量}

    上述过程中, 我们发现竖的矢量, 横的矢量, 矩阵, 所有这些对象坐标转化不同, 
    我们为了描述一个对象的坐标变化, 引入张量的概念。先把上面的方程写成分量的形式
    (我们把行标记作下标, 列标记作上标)。

    \begin{equation}
        v^\prime_i = \sum_{j=1}^3 U^j_i v_j
    \end{equation}

    \begin{equation}
        v^{\prime i} = \sum_{j=1}^3 U^i_j v^j
    \end{equation}

    \begin{equation}
        R^{\prime i}_j = \sum_{m=1}^3 \sum_{n=1}^3 U^i_m U^n_j R^m_n
    \end{equation}

    略去重复两遍指标的$\Sigma$(称为Einstein notation), 写成(以下都略去):

    \begin{equation}
        v^\prime_i = U^{j^\prime}_i v_j
    \end{equation}

    \begin{equation}
        v^{\prime i} = U^i_{j^\prime} v^j
    \end{equation}

    \begin{equation}
        R^{\prime i}_j = U^{i^\prime}_m U^n_{j^\prime} R^m_n
    \end{equation}

    事实上, $U$是:

    \begin{equation}
        U^j_i = \frac{\partial x_i}{\partial x_j^\prime}
    \end{equation}

    如果一个量有$3^{p+q}$个分量, 而且在坐标变换$U$下, 满足

    \begin{equation}
        T^{\prime i_1 \cdots i_p}_{j_1 \cdots j_q} = U^{i_1^\prime}_{n_1} \cdots U^{i_p^\prime}_{n_p} T^{n_1 \cdots n_p}_{m_1 \cdots m_q} U^{m_1}_{j_1^\prime} \cdots U^{m_q}_{j_q^\prime}
    \end{equation}

    那么就称这个量为$(p,q)$型张量。如果$p=q=0$, 那么就是标量, 如果$q=1$, 那么就是矢量, 如果$p=q=1$, 那么就对应了一个矩阵
    (概念上请注意, 矩阵仅仅是数, 而张量是定义于空间上的一个量, 在一定坐标下一定表达)。

    张量在之后的学习中, 包括场论, 广相等都是很重要的概念, 所有物理方程都是张量等式。

    如果一个$(0,p)$型张量满足:

    \begin{equation}
        T_{i_1 \cdots i_{m-1} i_n i_{m+1} \cdots i_{n-1} i_m i_{n+1} \cdots i_p} = 
        - T_{i_1 \cdots i_p}
    \end{equation}

    那么就称这个张量为全反对称张量, 全反对称张量和微分结构有关, 
    微分结构可以写成若干个$dx_{i_1} \wedge dx_{i_2} \wedge \cdots \wedge dx_{i_p}$求和的形式, 并满足:

    \begin{equation}
        dx_{i_1} \wedge dx_{i_2} \cdots \wedge dx_{i_m-1} \wedge dx_{i_n} \wedge dx_{i_{m+1}} \cdots \wedge dx_{i_{n-1}} \wedge dx_{i_m} \wedge dx_{i_{n+1}} \cdots \wedge dx_{i_p} = 
        - dx_{i_1} \wedge dx_{i_2} \cdots \wedge dx_{i_p}
    \end{equation}

    $n$维空间上的$p$阶微分结构是$\frac{n!}{p!(n-p)!}$维的线性空间。
    在三维空间中, 可以定义从二阶微分结构到一阶微分结构的映射。
    \footnotemark
    \footnotetext{事实上, 给定度量$g$后可以定义$p$到$n-p$的映射。}

    \begin{eqnarray}
        dx_1 = dx_2 \wedge dx_3 \\
        dx_2 = dx_3 \wedge dx_1 \\
        dx_3 = dx_1 \wedge dx_2
    \end{eqnarray}

    显然, 角动量就是全反对称张量。

    \subsection{坐标系变换}

    考虑三维正交坐标系$(x_1, x_2, x_3)$, 以及另一个正交坐标系$(x_1^\prime, x_2^\prime, x_3^\prime)$。

    \subsubsection{平动}

        平动是指坐标系的平移, 也就是说, 坐标系的原点在空间中的位置发生了变化, 但是坐标系的方向没有变化。

        \begin{equation}
            \begin{aligned}
                x_1^\prime = x_1 + s_1 \\
                x_2^\prime = x_2 + s_2 \\
                x_3^\prime = x_3 + s_3
            \end{aligned}
        \end{equation}

        两边求导有:

        \begin{equation}
            \begin{aligned}
                \frac{\dd x_1^\prime}{\dd t} = \frac{\dd x_1}{\dd t} + \frac{\dd s_1}{\dd t} \\
                \frac{\dd x_2^\prime}{\dd t} = \frac{\dd x_2}{\dd t} + \frac{\dd s_2}{\dd t} \\
                \frac{\dd x_3^\prime}{\dd t} = \frac{\dd x_3}{\dd t} + \frac{\dd s_3}{\dd t}
            \end{aligned}
        \end{equation}

        \begin{equation}
            \begin{aligned}
                \frac{\dd^2 x_1^\prime}{\dd t^2} = \frac{\dd^2 x_1}{\dd t^2} + \frac{\dd^2 s_1}{\dd t^2} \\
                \frac{\dd^2 x_2^\prime}{\dd t^2} = \frac{\dd^2 x_2}{\dd t^2} + \frac{\dd^2 s_2}{\dd t^2} \\
                \frac{\dd^2 x_3^\prime}{\dd t^2} = \frac{\dd^2 x_3}{\dd t^2} + \frac{\dd^2 s_3}{\dd t^2}
            \end{aligned}
        \end{equation}

        发现除了坐标系变换导致的$S(t)$的改变, $\frac{\dd S(t)}{\dd t}$和
        $\frac{\dd^2 S(t)}{\dd t^2}$都有改变, 称为惯性速度与惯性加速度。

    \subsubsection{转动}

        转动是指坐标基矢量的变换, 

        \begin{equation}
            \begin{bmatrix}
                x_1^\prime \\
                x_2^\prime \\
                x_3^\prime
            \end{bmatrix} = \Omega \begin{bmatrix}
                x_1 \\
                x_2 \\
                x_3
            \end{bmatrix}
        \end{equation}

        两边求导有:

        \begin{equation}
            \begin{bmatrix}
                \frac{\dd x_1^\prime}{\dd t} \\
                \frac{\dd x_2^\prime}{\dd t} \\
                \frac{\dd x_3^\prime}{\dd t}
            \end{bmatrix} = \Omega \begin{bmatrix}
                \frac{\dd x_1}{\dd t} \\
                \frac{\dd x_2}{\dd t} \\
                \frac{\dd x_3}{\dd t}
            \end{bmatrix} + \frac{\dd \Omega}{\dd t} \begin{bmatrix}
                x_1 \\
                x_2 \\
                x_3
            \end{bmatrix} = \Omega \begin{bmatrix}
                \frac{\dd x_1}{\dd t} \\
                \frac{\dd x_2}{\dd t} \\
                \frac{\dd x_3}{\dd t}
            \end{bmatrix} + \Omega \omega \begin{bmatrix}
                x_1 \\
                x_2 \\
                x_3
            \end{bmatrix}
        \end{equation}

        \begin{equation}
            \begin{bmatrix}
                \frac{\dd^2 x_1^\prime}{\dd t^2} \\
                \frac{\dd^2 x_2^\prime}{\dd t^2} \\
                \frac{\dd^2 x_3^\prime}{\dd t^2}
            \end{bmatrix} = \Omega \begin{bmatrix}
                \frac{\dd^2 x_1}{\dd t^2} \\
                \frac{\dd^2 x_2}{\dd t^2} \\
                \frac{\dd^2 x_3}{\dd t^2}
            \end{bmatrix} + \frac{\dd^2 \Omega}{\dd t^2} \begin{bmatrix}
                x_1 \\
                x_2 \\
                x_3
            \end{bmatrix} + 2\frac{\dd \Omega}{\dd t} \begin{bmatrix}
                \frac{\dd x_1}{\dd t} \\
                \frac{\dd x_2}{\dd t} \\
                \frac{\dd x_3}{\dd t}
            \end{bmatrix}
        \end{equation}

        假定角速度$\omega$不变, 则

        \begin{equation}
            \begin{bmatrix}
                \frac{\dd^2 x_1^\prime}{\dd t^2} \\
                \frac{\dd^2 x_2^\prime}{\dd t^2} \\
                \frac{\dd^2 x_3^\prime}{\dd t^2}
            \end{bmatrix} = \Omega \begin{bmatrix}
                \frac{\dd^2 x_1}{\dd t^2} \\
                \frac{\dd^2 x_2}{\dd t^2} \\
                \frac{\dd^2 x_3}{\dd t^2}
            \end{bmatrix} + \Omega \omega^2 \begin{bmatrix}
                x_1 \\
                x_2 \\
                x_3
            \end{bmatrix} + 2\Omega \omega \begin{bmatrix}
                \frac{\dd x_1}{\dd t} \\
                \frac{\dd x_2}{\dd t} \\
                \frac{\dd x_3}{\dd t}
            \end{bmatrix}
        \end{equation}

        上述$\Omega\omega^2\thicksim$, $2\Omega\omega\thicksim$两项被称为
        离心加速度和科里奥利加速度, 在学习牛顿三定律后会有更加全面的认识。

    \subsection{习题}

    \begin{exercise}
        假设物体以$\omega$为角速度, 从$I$处开始旋转, 求证:
        \begin{equation}
            \Omega = MatrixExp(\omega t) = I + \omega t + \frac{1}{2}\omega^2 t^2 + \frac{1}{6} \omega^3 t^3 \cdots
        \end{equation}
    \end{exercise}

    \begin{exercise}
        求证:
        \begin{equation}
            MatrixExp(A) MatrixExp(B) = MatrixExp(A + B + \frac{1}{2} [A, B])
        \end{equation}
        其中$[A, B] = AB - BA$。
    \end{exercise}

    \begin{exercise}
        定义矩阵
        \begin{equation}
            \begin{aligned}
                \Sigma_1 = \begin{bmatrix}
                    0 & 0 & 0 \\
                    0 & 0 & 1 \\
                    0 & -1 & 0
                \end{bmatrix} \\
                \Sigma_2 = \begin{bmatrix}
                    0 & 0 & -1 \\
                    0 & 0 & 0 \\
                    1 & 0 & 0
                \end{bmatrix} \\
                \Sigma_3 = \begin{bmatrix}
                    0 & 1 & 0 \\
                    -1 & 0 & 0 \\
                    0 & 0 & 0
                \end{bmatrix}
            \end{aligned}
        \end{equation}
        计算他们相互对易$[A,B]$的结果。
    \end{exercise}
    
    \begin{exercise}
        定义椭球坐标系$(\zeta_1,\zeta_2,\zeta_3)$为下述方程的三个根。
        \begin{equation}
            \frac{x^2}{a^2+\zeta} + \frac{y^2}{b^2+\zeta} + \frac{z^2}{c^2+\zeta} = 1
        \end{equation}
        导出其坐标与切从的对应关系。
    \end{exercise}
\end{document}

% by bts
% rewriting chapter 1
% The first version is too mathy, and full of definitions
% plz use xelatex to compile
